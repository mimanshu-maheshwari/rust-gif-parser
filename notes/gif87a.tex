%%%%%%%%%%%%PREAMBLE%%%%%%%%%%%%
\documentclass{report}
\usepackage[utf8]{inputenc}
\usepackage[english]{babel}
\usepackage{amsfonts}
\usepackage{amssymb}
\usepackage{amsmath}
%\usepackage{amsthm}
%\usepackage{float}
\usepackage{graphicx}
\usepackage{enumerate}
\usepackage[margin=1in]{geometry}
%\usepackage{pgfplot}
\usepackage{fancyhdr}
\usepackage{tikz}

\title{GIF Encoder and Decoder \\ \large{Notes}}
\author{Mimanshu Maheshwari}

\pagestyle{fancy}
\fancyhead[l]{Mimanshu Maheshwari}
\fancyhead[r]{\today}
\fancyhead[c]{\thepage}
\fancyfoot[l]{Mimanshu Maheshwari}
\fancyfoot[r]{\today}
\fancyfoot[c]{\thepage}
\renewcommand{\headrulewidth}{0.2pt}
\setlength{\headheight}{15pt}


%%%%%%%%%%%%END%%%%%%%%%%%%



\begin{document}

\maketitle

\tableofcontents
%\listoftables

\chapter{GIF}
\section{Introduction}
Graphics Interchange Format (GIF) is a standard defining a mechanism for the storage and transmission of the storage and transmission of raster-based graphics information made in June 15, 1987 by CompuServe Incorporated. 

\chapter{GIF87a}

\section{General file Format}

\begin{table}[h]
	\begin{tabular}{|l|l|}
		\hline
		\textbf{Field} & \textbf{Repeat value} \\\hline 
		Gif Signature & No repeat \\ \hline
		Screen Descriptor & No repeat  \\ \hline
		Global Color Map &  No repeat \\ \hline
		Image Descriptor & repeated 1 to n times \\ \hline
		Local Color Map  & repeated 1 to n times \\ \hline
		Raster Data  & repeated 1 to n times \\ \hline
		Gif Terminator &  No repeat \\ \hline
	\end{tabular}
\end{table}

\section{Gif Signature}
Gif Signature consist of 
\section{Screen Descriptor}
\section{Global Color Map}
\section{Image Descriptor}
\section{Local Color Map}
\section{Raster Data}
\section{Gif Terminator}

\chapter{GIF89a}

\appendix

\end{document}
